\chapter{Charakterystyka ruchu pieszych}

\section{Points of intrest}
\label{sec:pointsOfInterest}
 Points
 
\section{Strefa prywatna}
\label{sec:strefaPryw}

Dla każdego pieszego definiuje się opisaną wcześniej strefę prywatna.
Im bliżej przeszkody lub innego agenta, tym pieszy czuje się mniej komfortowo i utrzymuje dystans od sąsiada zależny od konkretnej sytuacji. Strefa prywatna pomaga unikać kolizji w przypadkach nagłej zmiany prędkości przez innych uczestników ruchu.

\section{Czas relaksacji}
\label{sec:czasRelaksacji}

Piesi zmieniając kierunek swojej drogi potrzebują pewien (niewielki) czas na podjęcie decyzji. Z tego względu we wzorze został wprowadzony \textit{czas realaksacji}. Wartość przyjęta w symulacji to $0.5 sek$

\section{Czekający piesi}
\label{sec:czekajacyPiesi}

Czekający piesie są częstymi uczestnikami normalnego ruchu. Często zatrzymujemy się, aby porozmawiać przez telefon, zawiązać sznurowadła czy w oczekiwaniu na przyjazd windy. Czekający piesi mogą powodować korki \cite{6}. Modelowanie czekających pieszych ma dwa aspekty: reakcje  przechodzących obok pieszych na pozostającego w spoczynku oraz pozosrającego w spoczynku na poruszających się.

\section{Formowanie strug}
\label{sec:strugi}

