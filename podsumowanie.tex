\chapter{Podsumowanie}
\label{cha:podsumowanie}

\section{Potencjalne kierunki dalszych prac}
Temat symulacji ruchu pieszych jest bardzo rozległy. Obecne modyfikacje samego SFM pozwalają na implementację różnych zachowań pieszych. Model mógłby zostać wzbogacony o~zachowanie pieszych pod wpływam paniki, zagrożenia (np. związanego z pożarem) lub ruchem innych obiektów (np. samochodów). W modelu możnaby również bardziej zróżnicować pojedynczych agentów, jako że w świecie realnym mamy do czynienia z ludźmi starszymi, którzy poruszają się z~niższą prędkością, dziećmi czy osobami niepełnosprawnymi. Symulacja mogłaby zostać zaimplementowana przy użyciu wielu wątków. Pozwoliłoby to na pokazanie większej ilości detali i~częstosze wyszukiwanie ścieżki ruchu pieszego. Ponadto dla lepszej wizualizacji graficznej model mógłby zostać wzbogacony o~trzeci wymiar. Ciekawym wydaje się również implementacja liderów, którzy istotnie wpływają na zachowanie pieszych w~szczególności w~sytuacjach kryzysowych takich jak ewakuacja, tak jak zaproponowano w pracy (Guided Crowd Dynamics via Modified Social Force Model\cite{GuideCrowdDynViaModifiedSocialForceModel}). Zmodyfikowane modele Social Force pozwalają także na obliczanie ewentualnej ilości ofiar lub osób zranionych, co ma znaczenie w przypadku symulacji ewakuacji. 
Ciekawym wydaje się także zastosowanie logiki rozmytej \cite{modelingFuzzyLogic}.
Temat symulacji jest bardzo rozbudowany, może on zdaniem autora być kierunkiem badań niejednej pracy. 

\section{Wnioski}
\label{sec:wnioski}

Celem niniejszej pracy inżynierskiej była symulacja dynamiki ruchu pieszych. Po porównaniu istniejących modeli wybrany został Model Social Force ze względu na swoje możliwości oraz zastosowanie do problemu postawionego w pracy. Z~sukcesem zaimplementowany został sam model, algorytm wyszukiwania najkrótszej ścieżki, moduł unikania kolizji oraz symulacja graficzna.\\
%----------
Symulacja pozwala na modyfikowanie wielu parametrów takich jak ilość pieszych, rodzaje map, zakresy prędkości czy parametry samego SFM. Dzięki takim możliwościom można otrzymać bardzo ciekawe rezultaty i dogłębnie zbadać dynamikę ruchu pieszych.\\
%---------
Poprzez wielokrotne testy oraz odpowiedni dobór parametrów udało się uzyskać rezultaty odpowiadające rzeczywistości czego potwierdzeniem są przedstawione wykresy oraz ich zgodność z innymi symulacjami \cite{GuideCrowdDynViaModifiedSocialForceModel} i \cite{SocialForceSuwala}

Symulacja nie jest wolna od wad. Czasami zdarzają się zakleszczenia oraz występuje problem wydajnościowy przy większej ilości pieszych (powyżej 90), co uniemożliwia płynne wyświetlanie symulacji. W przypadku większej ilości pieszych występuje także problem z oscylacjami, jednakże jest to znany problem SFM \cite{oscillations}.

Stworzona symulacja może być świetną podstawą do dalszego rozwoju prac. Zdaniem autora aspektami wartymi szczególnie wartymi rozwinięcia jest implementacja różnych sytuacji takich jak ewakuacja pod wpływem pożaru lub innego zagrożenia oraz dalszy rozwój modelu do unikania kolizji pomiędzy pieszymi.

\section{Napotkane problemy}

Istnieje wiele publikacji poświęconych tematyce ruchu pieszych, jednakże większość z nich nie porusza całości problemu. Ciężko jest doszukać się kompleksowego wytłumaczenia wszystkich zjawisk, takich jak formowanie strug czy unikanie kolizji, w obrębie jednego zaproponowanego modelu.

W trakcie tworzenia pracy autor napotkał następujące problemy:

\begin{itemize}
\item wydajność algorytmu A* do wyznaczania optymalnych ścieżek przejścia. Implementacja samego algorytmu nie należy do najbardziej skomplikowanych. Ścieżki zostały łatwo wyznaczone, jednak problem pojawił się przy testach dla większej ilości agentów. Poprzez reprezentację tablicową dwóch zbiorów \textit{open set} oraz \textit{close set}, potrzebnych do działania algorytmu, czas wyszukiwania ścieżki na mapie z ilością węzłów ok. $480000$ wynosił ok. $10 sek$. Jest to czas wysoce odbiegający od potrzeb symulacji. Poprzez reprezentację zbiorów jako stos, czas ten skrócił się do ok. $100-200 ms$ dla takiej samej ilości węzłów. Czas wyszukiwania zależny jest w szczególności od ilości punków węzłowych, ale także od odległości między punktem początkowym i końcowym oraz ilości przeszkód.

\item w praktyce wszystkie publikacje dotyczące tematyki symulacji skupiają się w dużej mierze na opisie konkretnego modelu, nie dotykając przy tym kwestii jego praktycznego zastosowania. Dlatego jednym z największych problemów było zastosowanie wybranego modelu do symulacji.
\end{itemize}
