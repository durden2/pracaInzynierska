\chapter{Podsumowanie}
\label{cha:podsumowanie}

\section{Potencjalne kierunki dalszych prac}
Temat symulacji ruchu pieszych jest bardzo rozległy. Obecne modyfikacje samego SFM pozwalają na implementację różnych zachowań pieszych. Model mógłby zostać wzbogacony o zachowanie pieszych pod wpływam paniki, zagrożenia (np. związanego z pożarem) lub ruchem innych obiektów (np. samochodów). W modelu możnaby również bardziej zróżnicować pojedynczych agnetów, wszak w świecie realnym mamy do czynienia z ludźmi starszymi, którzy poruszają się z niższą prędkością, dzieci czy osoby niepełnosprawne. Symulacja mogłaby zostać zaimplementowana używająć wielu wątków. Pozwoliłoby to na pokazanie większej ilości detali i częściej wyszukiwać ścieżkę ruchu pieszego. Ponadto dla lepszej wizualizacji graficznej model młgłby zostać wzbogacony o trzeci wymiar. Ciekawym wydaje się również implementacja liderów, którzy istotnie wyływają na zachowanie pieszych w szczególności w sytuacjach kryzysowych takich jak ewakuacja, tak jak zaproponowano w pracy Guided crowd dynamics via modified social force model\cite{GuideCrowdDynViaModifiedSocialForceModel}. Zmodyfikowane modele Social Force pozawalają także na obliczanie ewentualnej ilości ofiar lub osób zranionych, co ma znaczenie w przypaku symulacji ewakuacji. 
Ciekawym wydaje się także zastosowanie logiki rozmytej \cite{modelingFuzzyLogic}.
Temat symulacji jest bardzo rozbudowany może on zdaniem autora być kierunkiem badań nie jednej pracy. 

\section{Wnioski}
\label{sec:wnioski}

W pracy zostało...


\section{Napotkane problemy}

Istnieje wiele publikacji poświęconych tematyce ruchu pieszych, jednakże wiele z nich nie porusza całości problemu. Ciężko jest doszukać się kompleksowego wytłumaczenia wszystkich zjawisk, takich jak formowanie strug czy unikanie kolizji, w obrębie jednego zaproponowanego modelu.

W trakcie tworzenia pracy autor napotkał następujące problemy:

\begin{itemize}
\item wydajność algorytmu A* do wyznaczania optymalnych ścieżek przejścia. Implementacja samego algorytmu nie należy do najbardziej skomplikowanych. Ścieżki zostały łatwo wyznaczone, jednak problem pojawił się przy testach dla większej ilości agentów. Poprzez reprezentację tablicową dwóch zbiorów \textit{open set} oraz \textit{close set} potrzebnych do działania algorytmu, czas wyszukiwania ścieżki na mapie z ilością węzłów około $480000$ wynosił ok $10 sek$. Jest to czas wysoce odbiegający od potrzeb symulacji. Poprzez reprezentację zbiorów jako stos, czas ten skrócił się do koło $100-200 ms$ dla takiej samej ilości węzłów. Czas wyszukiwania zależny jest w szczególności od ilości punków węzłowych, ale także od odległości między punktem początkowym i końcowym oraz ilości przeszkód.

\item w praktyce wszystkie publikacje dotyczące tematyki symulacji skupiają się w dużej mierze na opisie konkretnego modelu, nie dotykając przy tym kwestii jego praktycznego zastosowania. Dlatego jednym z największych problemów było zastosowanie wybranego modelu do symulacji.
\end{itemize}
