\chapter{Wprowadzenie}
\label{cha:wprowadzenie}

Zachowanie tłumu badane jest od przeszło trzech dekad. Na samym początku badania były traktowane w ramach ciekawostki. Wraz z nowatorskimi pracami Helbinga .... . W ostatnich czasach symulacje ruchu pieszych zyskały na popularności. Głównie za sprawą analogii do zachowania gazów oraz płynów \cite{SforceModelForPedDyn}. Nie bez znaczenia jest także łatwość uzyskania parametrów oraz wartości potrzebnych do symulacji. Wartości takie jak predkość $\vec{v_{\alpha}}$ czy położenie $\vec{r_{\alpha}}$ danego pieszego $\alpha$ są łatwe do obliczenia, ale także do skalibrowania z danami empirycznymi. Modelowanie ruchu pieszych odgrywa ponadto dużą rolę w projektowaniu, dostarcza wielu informacji użytecznych podczas planowania miejsc użyteczności publicznej

Jesteśmy obecnie świadkami rozrostu miast, budowy kompleksów sportowych czy galerii handlowych. Wszystkie te miejsca są nieodłącznie związane z tłumami przewijających się przez nie osób. W związku z rosnącą gęstością zaludnienia oraz wzrostem zagorzeń takich jak terroryzm [Przypis?] tworzenie symulacji ewakuacji nabrało większego znaczenia. Dzięki zasymulowaniu zachowania tłumu możemy łatwiej utworzyć schematy opuszczenia budynków podczas zagrożenia minimalizując szkody oraz ofiary. Symulacje pozwalają także na lepsze rozładowanie ruchu drogowego w miastach o rosnącej gestości zaludnienia. Symulacje mogą mieć wielorakie zastosowanie, począwszy od ewakucaji ludności poprzez zachowania w centrach handlowych kończąc na ruchu drogowym. Na przejściach dla pieszych w Japonii ginie 30\% osób uczestniczących w wypadkach drogowych \cite{AMSFMfPBSaSC}, a w Niemczech odsetek ten wynosi 15\% \ [German instigute for econeomic research 2010]. Zgodnie z danymi organizacjie Fire Administration w Stanach Zjednoczonych \cite{Asfemwle} w roku 2007 3430 osób zmarło w pożarach oraz blisko 18 tysięcy zostało rannych. Biorąc pod uwagę te dane łatwo dojść do wniosku jakie korzyści płyną z symulacji ruchu pieszych.

%---------------------------------------------------------------------------

\section{Cele pracy}
\label{sec:celePracy}

Celem poniższej pracy jest zasymulowanie ruchu pieszych.

%---------------------------------------------------------------------------

\section{Zawartość pracy}
\label{sec:zawartoscPracy}

W rodziale~\ref{cha:wprowadzenie} przedstawiono podstawowe informacje dotyczące symulacji komputerowych z ruchem pieszych. Kolejny rozdział poświęcony jest porównaniu istniejących rozwiązań pomagających takie symulacje zaimplemnetować. Rozdział ...

\section{Zastosowanie symulacji komputerowych}
\label{sec:ZastosowanieSymulacji}


















