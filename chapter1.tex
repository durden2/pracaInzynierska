\chapter{Wprowadzenie}
\label{cha:wprowadzenie}

Jesteśmy obecnie świadkami dynamicznego rozrostu miast, budowy dużych kompleksów sportowych czy galerii handlowych. Wszystkie te miejsca są nieodłącznie związane z tłumami przewijających się przez nie osób.~W związku~z rosnącą gęstością dynamiki tłumu oraz wciąż utrzymującym zagrożeniami terrorystycznymi \cite{terrorism} tworzenie symulacji nabrało większego znaczenia. W ostatnich dekadach ilość wypadków związanych ze złym planowaniem ewakuacji wzrosła. Katastrofy takie jak tragedia w Hillsborough w roku 1989 (96 ofiar) \cite{hillsborough} czy wybuch paniki na Love Parade w Duisburgu w roku 2010 (21 ofiar) \cite{lovedisaster} pokazują, że efektywność ewakuacji stała się kluczowym aspektem bezpieczeństwa~w miejscach publicznych takich jak stadiony, stacje metra czy lotniska. Symulacje mogą mieć wielorakie zastosowanie, począwszy od wspomnianej ewakuacji ludności, poprzez filmografię, kończąc na badaniu skutków katastrof. Dzięki symulowaniu zachowania tłumu możemy łatwiej utworzyć plany ewakuacyjne na wypadek zagrożenia minimalizując szkody, a przede wszystkim ofiary. Pozwalają one także na lepsze zarządzanie ruchem drogowym~w miastach o rosnącej gęstości zaludnienia.
%-----------
Podróże piesze są bardzo częstym sposobem przemieszczania się w przestrzeni miejskiej. Na przejściach dla pieszych w Japonii ginie 30\% osób uczestniczących w wypadkach drogowych \cite{AMSFMfPBSaSC},~w Niemczech odsetek ten wynosi 15\% \ [German instigute for econeomic research 2010], zaś w Polsce odsetek ofiar śmiertelnych w ruchu drogowym wśród pieszych wynosi 34\% \cite{metodologiaZachowan}. Zgodnie z danymi organizacji Fire Administration w Stanach Zjednoczonych w roku 2007 3430 osób zmarło w pożarach oraz blisko 18 tysięcy zostało rannych \cite{Asfemwle}. Biorąc pod uwagę te dane, łatwo dojść do wniosku, jakie korzyści mogą płynąć z symulacji zachowania pieszych.
%-----------
Zachowanie tłumu badane jest od przeszło trzech dekad. Na samym początku badania były traktowane~w ramach ciekawostki. Ostatnie lata przyniosły wiele zmian w dziedzinie modelowania dynamiki tłumu. Wraz z nowatorskimi pracami Helbinga \cite{SforceModelForPedDyn} większą popularność zyskały modele mikroskopowe, takie jak Model Social Force, które wyparły dotychczas stosowane modele makroskopowe bazujące na zasadach hydrodynamiki  \cite{SforceModelForPedDyn}. Modelowanie ruchu pieszych odgrywa dużą rolę w projektowaniu, dostarcza wielu informacji użytecznych podczas planowania miejsc użyteczności publicznej. Znajomość problemów mogących wystąpić podczas ewakuacji oraz sytuacji konfliktowych pomiędzy pieszymi jest kluczowa w projektowaniu nowych budynków. W przypadku małych budynków łatwo można przeprowadzić próbne ewakuacje, jednakże kiedy w grę wchodzą duże kompleksy, nie jest to możliwe. Nawet próba z małą ilością osób nie pokaże skali problemu. Nie jesteśmy wówczas~w stanie zbadać, jak na ewakuację wpłynie rosnąca gęstość tłumu.
%-----------------
W pracy autor dokonuje próby symulacji zachowania tłumu oraz zbadania miejsc o dużej gęstości. Porównane zostają wyniki dla różnej ilości pieszych oraz w różnych pomieszczeniach nakreślając,~w jaki sposób na zachowanie tłumu wpływa rozmieszczenie ścian oraz wyjść~w budynkach.


%---------------------------------------------------------------------------

\section{Cele pracy}
\label{sec:celePracy}

Głównym celem pracy jest implementacja oraz przeprowadzenie symulacji zachowania pieszych~z użyciem metody Social Force. Symulacja ruchu pieszych skupia się na dużych zgromadzeniach ludzi~w małych przestrzeniach. Zaproponowany model pozwala na opisanie każdej indywidualnej jednostki~z osobna, biorąc pod uwagę jej personalne cechy, takie jak prędkość ruchu czy masę. Symulacja bierze pod uwagę także aspekty psychologiczne oraz socjologiczne, jakie można nakreślić, badając zachowanie tłumu.

%---------------------------------------------------------------------------

\section{Zawartość pracy}
\label{sec:zawartoscPracy}
W rozdziale \ref{cha:wprowadzenieTeoretyczne} nakreślone zostają teoretyczne aspekty symulacji oraz porównania istniejących rozwiązań pomagających takie symulacje zaimplementować, rozdział \ref{cha:wyznaczenieSciezki} skupia się na opisie sposobu wyznaczenia najkrótszej ścieżki jaką podąża pieszy. Rozdział \ref{cha:OpisSocialForce} opisuje Model Social Force używany w prac, rozdział \ref{cha:charakterystykaRuchu}opisane zostają zjawiska towarzyszące ruchowi pieszych. Rozdział \ref{cha:schematy}: przedstawiony zostaje dokładny schemat działania symulacji, rozdział \ref{cha:testy}: podsumowanie wykonanych testów, rozdział \ref{cha:podsumowanie}: podsumowanie wykonanej pracy. \\


\section{Zastosowanie symulacji komputerowych}
\label{sec:ZastosowanieSymulacji}

Symulacje komputerowe w dzisiejszych czasach mają bardzo szerokie zastosowanie. Pozwalają dokonać analizy procesów, których odtworzenie w świecie rzeczywistym byłoby bardzo czasochłonne lub takie, które jest trudne do realizacji (np. ewakuacje budynków). Ich zastosowanie pozwala na uniknięcie trudnych do przewidzenia błędów. Dzięki swoim możliwościom, znalazły one zastosowanie w dziedzinach takich jak:

\begin{itemize}
\item Symulacje statków powietrznych (np. do szkolenia przyszłych pilotów),
\item ekonomia i biznes (systemy kolejkowe, zarządzanie zapasami),
\item nauki społeczne (dynamika populacji, prognozowanie podziału miejsc w parlamencie),
\item nauki inżynieryjne i przyrodnicze (meteorologia, wytrzymałość konstrukcji),
\item komputerowe gry symulacyjne,
\item kryminalistyka (np. rekonstrukcja przebiegu wydarzeń wypadków).
\end{itemize}

\section{Problemy związane z symulacjami komputerowymi}
\label{sec:ProblemyzSymulacjami}

W przeciągu stosunkowo krótkiej historii informatyki możemy zaobserwować ogromny wzrost dostępnych mocy obliczeniowych komputerów. Zgodnie z Prawem Moore'a liczba tranzystorów podwaja się w przeciągu 24 miesięcy \cite{mooreslaw}. Ma to duży wpływ na rozwój symulacji komputerowych, a w szczególności odzwierciedlenia wszystkich detali. Jednakże ciągły wzrost skomplikowania systemów oraz potrzeba coraz to większej dokładności rezultatów powodują, że symulacje komputerowe napotykają wile wyzwań. Wielość czynników wpływających na złożoność problemów symulacji nie tylko na poziomie koncepcyjnym oraz wydajnościowym, ale~i implementacyjnym sprawia główną trudność związaną~z symulacjami komputerowymi. Istnieją sytuacje w realnym świecie, które nie mogą zostać zwizualizowane nawet na najszybszych dostępnych komputerach \cite{simulationLimits}. Modele używane w symulacjach zawsze opierają się na jakiś przybliżeniach świata realnego, kiedy w nim samym złożoność systemów jest nieskończona \cite{simulationInGeneral}. Do stworzenia symulacji potrzebna jest współpraca uczonych z wielu dziedzin, co powoduje trudność w opisaniu potrzebnych modeli. Modeli nadal brakuje \cite{simulationChallenges}. Po zamodelowaniu danego procesu podejmuje się zazwyczaj kroki w celu weryfikacji modelu, aby w możliwie największym stopniu odpowiadał rzeczywistości, co powoduje kolejną trudność. Kończąc symulacje jak każdy inny program komputerowy podatne są na błędy co może w pewien sposób ograniczyć ich zastosowanie.
















