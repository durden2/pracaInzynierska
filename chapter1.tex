\chapter{Wprowadzenie}
\label{cha:wprowadzenie}

Zachowanie tłumu badane jest od przeszło trzech dekad. Na samym początku badania były traktowane w ramach ciekawostki. Wraz z nowatorskimi pracami Helbinga .... . W ostatnich latach zagadnienie to zyskuje coraz większe znaczenie. Jesteśmy obecnie świadkami rozrostu miast, budowy kompleksów sportowych czy galerii handlowych. Wszystkie te miejsca są nieodłącznie związane z tłumami przewijających się przez nie osób. W związku z rosnącą gęstością zaludnienia oraz wzrostem zagoreń takich jak terroryzm [jakis przypis] tworzenie symulacji ewakuacji nabrało większego znaczenia. Dzięki zasymulowaniu zachowania tłumu możenmy łatwiej utworzyć schematy opuszczenia bynków podczaz zagorżenia minimaliusjąc szkody oraz ofiary. Symulacje pozwalają także na lepsze rozładowanie ruchu drogowego w miastach o roznącej ilości zaludnienia.

Symulacje mogą mieć wielorakie zastosowanie, począwszy od ewakucaji ludności poprzez zachowania w centrach handlowych kończąc na ruchu drogowym. Na przejściach dla pieszych w Japonii ginie 30\% osób uczestniczących w wypadkach drogowych \cite{AMSFMfPBSaSC}, a w Niemczech odsetek ten wynosi 15\% \ [German instigute for econeomic research 2010]. Zgodnie z danymi organizacjie Fire Administration w Stanach Zjednoczonych \cite{Asfemwle} w roku 2007 3430 osób zmarło w pożarach oraz blisko 18 tysięcy zostało ranych.

%---------------------------------------------------------------------------

\section{Cele pracy}
\label{sec:celePracy}

Celem poniższej pracy jest zapoznanie studentów z systemem \LaTeX~w zakresie umożliwiającym im samodzielne, profesjonalne złożenie pracy dyplomowej w systemie \LaTeX.

%---------------------------------------------------------------------------

\section{Zawartość pracy}
\label{sec:zawartoscPracy}

W rodziale~\ref{cha:pierwszyDokument} przedstawiono podstawowe informacje dotyczące struktury dokumentów w \LaTeX u. Alvis~\cite{Alvis2011} jest językiem 


















