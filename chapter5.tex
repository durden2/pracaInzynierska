\chapter{Wyznaczenie ścieżki}
\label{wyznaczenieSciezki}

Wyznaczenie najkrótszych ścieżek w grafie jest jednym z podstawowych problemów w teorii grafów. Algorytmy wyszukiwania ścieżek mają wielorakie zastosowanie począwszy od wyznaczenia najkrótszych tras na mapie, poprzez przesyłanie wiadomości przez sieć routerów, kończąc na wyznaczaniu połączeń lotniczych o najmniejszym koszcie. Wynikiem działania algorytmów wyznaczania ścieżek, jest uporządkowany zbiór wierzchołków, którymi należy kolejno podążać, aby dotrzeć do wyznaczonego wcześniej celu.

Algorytmów wyszukiwania ścieżek jest bardzo wiele, najważniejsze z nich to:

\begin{itemize}
\item Algorytm Dijkstry - przykład algorytmu zachłannego. Jeden z najbardziej rozpowszechnionych algorytmów w dziedzinie przeszukiwania ścieżek. Jego złożoność obliczeniowa rośnie w miarę wzrostu punktów węzłowych,
\item Algorytm A* - algorytm wykorzystany w pracy. Jest to rozszerzona wersja algorytmu Djikstry. Dzięki zastosowaniu heurystyki skraca się czas obliczeń,
\item Algorytm Bellmana-Forda - ma zastosowanie, kiedy niektóre krawędzie w grafie mają ujemne wagi
\item Algorytm Floyda-Warshalla - pozwala na odnalezienie najkrótszych ścieżek pomiędzy wszystkimi parami wierzchołków w grafie,
\item Przeszukiwanie wszerz BFS - najprostszy z algorytmów, nie uwzględnia wag ścieżek,
\item Przeszukiwanie wgłąb DFS
\end{itemize}

Spośród wymienionych algorytmów implementacji symulacji został użyty algorytm A*. Gwarantuje on zawsze znalezienie optymalnej ścieżki jeśli tylko istnieje. Nie jest on też zbytecznie złożony obliczeniowo