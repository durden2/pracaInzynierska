\chapter{Wyznaczenie ścieżki}

Wyznaczenie najkrótrzych ścieżek w grafie jest jednym z podstawowych problemów w teorii grafów. Algorytmy wyszukiwania ścieżek mają wielorakie zastosowanie począwszy od wyznaczenia najkrótszych tras na mapie poprzez przesyłania wiadomości przez sieć routerów kończąc na wyznaczeniu połączeń lotniczych o najmniejszym koszcie. Wynikiem działania algorytmów jest uporządkowany zbiór wierzchołków, które musimy kolejno przejść, aby dotrzeć do wyznaczonego wcześniej celu.

Algorytmów wyszukiwania ścieżek jest bardzo wiele, najważniejsze z nich to:

\begin{itemize}
\item Algorytm Djikstry - jest przykładem algorytmu zachłannego. Jeden z najbardziej rozpowszechnionych algorytmów do przeszukiwania ścieżek. Jego złożoność obliczeniowa rośnie w miarę wzrostu punktów węzłowych.
\item A*
\item Algorytm Fleury'ego
\item Bellmana-Forda - ma zastosowanie, kiedy niektóre krawędzie w grafie mają ujemne wagi
\item Algorytm Floyda-Warshalla - 
\item Algorytm A*
\item Przeszukiwanie wszrze BFS
\item Przeszukiwanie wgłąb DFS
\end{itemize}