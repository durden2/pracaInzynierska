\chapter{Wprowadzenie teoretyczne}
\label{cha:wprowadzenieTeoretyczne}

Z pozoru zachowanie pieszych może wydawać się chaotyczne oraz trudne do przewidzenia. Bazując jednak na badaniach i obserwacjach takie zachowania mają miejsce tylko w skrajnych przypadkach. W codziennym życiu okazuje się, że model do opisu zachowania tłumu może być w dość prosty sposób opisany, głównie dzięki prawdopodobieństwom jakie mogą zostać nakreślone w duży populacjach ludzi. 


\section{Systemy modelowania ruchu pieszych}
\label{sec:modeleSymulacji}

Poniżej przedstawiam różne modele symulacji ruchu pieszych

\subsection{gas-kinetic pedestrian model}

W codziennym życiu okazuje się, że model do opisu zachowania tłumu może być w dość prosty sposób opisany, głównie dzięki prawdopodobieństwom jakie mogą zostać nakreślone w duży populacjach ludzi. 

\subsection{Automaty komórkowe}

Coś o automatach komórkowych
%---------------------------------------------------------------------------

\subsection{Social Force model}

Model Social Force zakłada, że piesi w ruchu mogą zostać w prosty sposób opisani za pomocą sił. Siły te pochodzą nie tylko z oddziaływań konkretnego pieszego na otoczenie, ale także z otoczenia na danego pieszego. Wartość, zwrot oraz kierunek finalnej siły jest składową wszystkich sił działających na danego pieszego. Dotychczasowe symulacje komputerowe pokazują, że Model Social Force, pomimo swojej prostoty bardzo realistycznie oddaje rzeczywiste zachowanie się tłumu.

\section{Wybór modelu}
\label{sec:wyborModelu}

Wybrano model symulacji pieszych bla bla


















