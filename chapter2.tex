\chapter{Wprowadzenie teoretyczne}
\label{cha:wprowadzenieTeoretyczne}

Informatyka nie jest jedyną dziedziną zajmującą się badaniem ruchu pieszych. Na powstanie modeli mają wpływ także prace uczonych z takich dziedzin jak psychologia, socjologia czy architektura. Problem zachowania tłumu jest skomplikowany i dopiero połączenie tych wszystkich dziedzin daje początek realnemu odzwierciedleniu na ekranie komputera. \\
Z pozoru zachowanie pieszych może wydawać się chaotyczne oraz trudne do przewidzenia. Bazując na badaniach i obserwacjach \cite{SforceModelForPedDyn}, dzieje się tak tylko w skrajnych przypadkach.~W codziennych sytuacjach zachowanie tłumu jest łatwe do przewidzenia, a model do jego opisu można skonstruować w oparciu o prawdopodobieństwo. Człowiek ma tendencję do podejmowania decyzji na bazie wypracowanej wiedzy na temat otaczającego go środowiska. Oznacza to, że reakcje na innych pieszych oraz przeszkody mogą być w łatwy sposób przewidziane. Analogią do takiego zachowania mogą być reakcje profesjonalnego kierowcy wyścigowego, który na sytuacje drogowe reaguje niemal automatycznie.\\

Oczywiście nie jest to prawdą w każdej sytuacji. Przykładowo dzieci oraz turyści poruszają się w inny sposób, ze względu na nieznajomość terenu i wynikający z niej brak strategii poruszania się. Jednakże dla potrzeb symulacji nie potrzeba dokładnych informacji~o każdym~z pieszych, w zupełności wystarcza statystyczna wartość konkretnych zachowań.

W tym rozdziale zostają przedstawione obecnie dostępne modele. Każdy~z modeli ma swoje zalety oraz wady, wpływają na to cechy takie jak złożoność obliczeniowa, złożoność implementacji oraz oczywiście  otrzymane rezultaty.

\section{Klasyfikacja modeli symulacji ruchu pieszych}
\label{sec:klasyfikacja}

\subsection{Modele makroskopowe oraz mikroskopowe}

Modele makroskopowe pokazują w głównej mierze dynamikę gęstości oraz prędkości całego tłumu. W tym celu używane są istniejące już modele fizyczne takie jak dynamika płynów oraz modele kolejkowe \cite{relativeVelocity}, które zostają odpowiednio dostosowane do potrzeb symulacji. Przykładem może być hydrodynamiczny model Paulusa opierający się na równaniach przepływu \cite{ArchitekturaModelowania}. Modele te nie biorą pod uwagę indywidualnych zachowań jednostki. Podejście makroskopowe, ze względu na odzwierciedlanie całej populacji, sprawdza się w praktyce tylko w wąskim wachlarzu zastosowań. Jako zaletę podejścia makroskopowego możemy~z pewnością wskazać mniejszą ilość obliczeń potrzebną do uzyskania pożądanego efektu. \\

Modele mikroskopowe, w przeciwieństwie do modeli makroskopowych, biorą pod uwagę zachowanie konkretnej jednostki. Badane są interakcje pomiędzy pieszymi,~z przeszkodami oraz otaczającą rzeczywistością. Modele te pozwalają na uzyskanie efektu bardziej odpowiadającego realnemu zachowaniu tłumu. Jednakże wraz ze wzrostem odwzorowania detali, wzrasta również złożoność systemu oraz zwiększa się złożoność obliczeń, co z pozoru może skutkować gorszą wydajnością. Jednak przy dostępnej obecnie mocy obliczeniowej nawet standardowych komputerów, nie jest to jednak aż tak istotne. Przykładem modelu mikroskopowego może być Automat Komórkowy \textit{Cellular Automata} lub Model Social Force (SFM). 

\subsection{Modele ciągłe i dyskretne}

W modelach mikroskopowych możemy wyodrębnić dwie podgrupy: modele dyskretne oraz ciągłe. Modele dyskretne cechują się zmianą parametrów stanu w konkretnych interwalach czasowych; przyjmują określone wartości dla określonych argumentów i tylko dla nich.~W modelach ciągłych stan ulega zmianie przez cały czas działania - może przyjmować dowolną wartość z całego przedziału. W związku z tym reprezentacja danych w modelach ciągłych jest bardziej realistyczna, czego kosztem jest dłuższy czas obliczeń. \\

Jednym z przykładów modelu dyskretnego może być automat komórkowy. Uniwersalność automatów komórkowych  spowodowała, że znajdują one zastosowanie także w dziedzinie symulacji. Automat komórkowy jest modelem matematycznym, który specyfikuje siatka komórek, zbiór stanów, jakie mogą one przyjmować oraz reguły określające stan komórki w chwili $t + 1$. Stan danej komórki zależny jest od stanu komórek~z nią sąsiadujących w chwili $t$. Możemy wyróżnić kilka rodzajów automatów \cite{modelowanieDynamikiTlumu}:

\begin{itemize}
\item klasyczne automaty; reguły określające kolejne stany komórek biorą pod uwagę tylko relacje lokalne
\item globalne automaty komórkowe; reguły obejmują obszar całej siatki,
\item poszerzone automaty komórkowe; reguły przejścia obejmują komórki lokalne oraz wybrane komórki z całego obszaru siatki.
\end{itemize}

Dla stworzenia modeli mikroskopowych używa się poszerzonych automatów komórkowych nazywanych także niehomogenicznymi automatami komórkowymi \cite{modelowanieDynamikiTlumu}. Niestety jednorodność tej metody uniemożliwia modelowanie bardziej skomplikowanych procesów\cite{FormalizacjaAutomatów}. Dopiero zastosowane w nich różnego typu komórek oraz zróżnicowanie funkcji przejścia na siatce, pozwala na różnicowanie zachowań poszczególnych jednostek. Wraz z rozwojem symulacji na automatach komórkowych można było dostrzec duże zmiany w bazowym modelu. Powstające symulacje zaczęły być klasyfikowane jako \textbf{sytemy agentowe}.


%---------------------------------------------------------------------------


















